% 2-1-correctness.tex

\documentclass[a4paper, justified]{tufte-handout}
% hw-preamble.tex

% geometry for A4 paper
% See https://tex.stackexchange.com/a/119912/23098
\geometry{
  left=20.0mm,
  top=20.0mm,
  bottom=20.0mm,
  textwidth=130mm, % main text block
  marginparsep=5.0mm, % gutter between main text block and margin notes
  marginparwidth=50.0mm % width of margin notes
}

% for colors
\usepackage{xcolor} % usage: \color{red}{text}
% predefined colors
\newcommand{\red}[1]{\textcolor{red}{#1}} % usage: \red{text}
\newcommand{\blue}[1]{\textcolor{blue}{#1}}
\newcommand{\teal}[1]{\textcolor{teal}{#1}}

\usepackage{todonotes}

% heading
\usepackage{sectsty}
\setcounter{secnumdepth}{2}
\allsectionsfont{\centering\huge\rmfamily}

% for Chinese
\usepackage{xeCJK}
\usepackage{zhnumber}
\setCJKmainfont[BoldFont=FandolSong-Bold.otf]{FandolSong-Regular.otf}

% for fonts
\usepackage{fontspec}
\newcommand{\song}{\CJKfamily{song}} 
\newcommand{\kai}{\CJKfamily{kai}} 

% To fix the ``MakeTextLowerCase'' bug:
% See https://github.com/Tufte-LaTeX/tufte-latex/issues/64#issuecomment-78572017
% Set up the spacing using fontspec features
\renewcommand\allcapsspacing[1]{{\addfontfeature{LetterSpace=15}#1}}
\renewcommand\smallcapsspacing[1]{{\addfontfeature{LetterSpace=10}#1}}

% for url
\usepackage{hyperref}
\hypersetup{colorlinks = true, 
  linkcolor = teal,
  urlcolor  = teal,
  citecolor = blue,
  anchorcolor = blue}

\newcommand{\me}[4]{
    \author{
      {\bfseries 姓名:}\underline{#1}\hspace{2em}
      {\bfseries 学号:}\underline{#2}\hspace{2em}\\[10pt]
      {\bfseries 评分:}\underline{#3\hspace{3em}}\hspace{2em}
      {\bfseries 评阅:}\underline{#4\hspace{3em}}
  }
}

% Please ALWAYS Keep This.
\newcommand{\noplagiarism}{
  \begin{center}
    \fbox{\begin{tabular}{@{}c@{}}
      请独立完成作业,不得抄袭。\\
      若得到他人帮助, 请致谢。\\
      若参考了其它资料,请给出引用。\\
      鼓励讨论,但需独立书写解题过程。
    \end{tabular}}
  \end{center}
}

% Each hw consists of four parts:
\newcommand{\beginrequired}{\hspace{5em}\section{作业 (必做部分)}}
\newcommand{\beginoptional}{\section{作业 (选做部分)}}

% for math
\usepackage{amsmath, mathtools, amsfonts, amssymb}
\newcommand{\set}[1]{\{#1\}}

% define theorem-like environments
\usepackage[amsmath, thmmarks]{ntheorem}

\theoremstyle{break}
\theorempreskip{2.0\topsep}
\theorembodyfont{\song}
\theoremseparator{}
\newtheorem{problem}{题目}[subsection]
\renewcommand{\theproblem}{\arabic{problem}}
\newtheorem{ot}{Open Topics}

\theorempreskip{3.0\topsep}
\theoremheaderfont{\kai\bfseries}
\theoremseparator{:}
\theorempostwork{\bigskip\hrule}
\newtheorem*{solution}{解答}

\theoremstyle{break}
\theorempostwork{\bigskip\hrule}
\theoremsymbol{\ensuremath{\Box}}
\newtheorem*{proof}{证明}

% \newcommand{\ot}{\blue{\bf [OT]}}

% for figs
\renewcommand\figurename{图}
\renewcommand\tablename{表}

% for fig without caption: #1: width/size; #2: fig file
\newcommand{\fig}[2]{
  \begin{figure}[htbp]
    \centering
    \includegraphics[#1]{#2}
  \end{figure}
}
% for fig with caption: #1: width/size; #2: fig file; #3: caption
\newcommand{\figcap}[3]{
  \begin{figure}[htbp]
    \centering
    \includegraphics[#1]{#2}
    \caption{#3}
  \end{figure}
}
% for fig with both caption and label: #1: width/size; #2: fig file; #3: caption; #4: label
\newcommand{\figcaplbl}[4]{
  \begin{figure}[htbp]
    \centering
    \includegraphics[#1]{#2}
    \caption{#3}
    \label{#4}
  \end{figure}
}
% for margin fig without caption: #1: width/size; #2: fig file
\newcommand{\mfig}[2]{
  \begin{marginfigure}
    \centering
    \includegraphics[#1]{#2}
  \end{marginfigure}
}
% for margin fig with caption: #1: width/size; #2: fig file; #3: caption
\newcommand{\mfigcap}[3]{
  \begin{marginfigure}
    \centering
    \includegraphics[#1]{#2}
    \caption{#3}
  \end{marginfigure}
}

\usepackage{fancyvrb}

% for algorithms
\usepackage[]{algorithm}
\usepackage[]{algpseudocode} % noend
% See [Adjust the indentation whithin the algorithmicx-package when a line is broken](https://tex.stackexchange.com/a/68540/23098)
\newcommand{\algparbox}[1]{\parbox[t]{\dimexpr\linewidth-\algorithmicindent}{#1\strut}}
\newcommand{\hStatex}[0]{\vspace{5pt}}
\makeatletter
\newlength{\trianglerightwidth}
\settowidth{\trianglerightwidth}{$\triangleright$~}
\algnewcommand{\LineComment}[1]{\Statex \hskip\ALG@thistlm \(\triangleright\) #1}
\algnewcommand{\LineCommentCont}[1]{\Statex \hskip\ALG@thistlm%
  \parbox[t]{\dimexpr\linewidth-\ALG@thistlm}{\hangindent=\trianglerightwidth \hangafter=1 \strut$\triangleright$ #1\strut}}
\makeatother

% for footnote/marginnote
% see https://tex.stackexchange.com/a/133265/23098
\usepackage{tikz}
\newcommand{\circled}[1]{%
  \tikz[baseline=(char.base)]
  \node [draw, circle, inner sep = 0.5pt, font = \tiny, minimum size = 8pt] (char) {#1};
}
\renewcommand\thefootnote{\protect\circled{\arabic{footnote}}} % feel free to modify this file
\title{第1讲: 算法问题与解题的正确性}
\me{林方浩}{231098078}{}{}
\date{截止时间:2024年3月3日} % or like 2019年9月13日
\newcommand{\lg}{\log_{10}}
\usepackage{bm}
\begin{document}
\maketitle
\noplagiarism % always keep this line

\beginrequired

%%%%%%%%%%%%%%%
\begin{problem}[TC Problem $2-1$]
\end{problem}

\begin{solution}
% 填写你的答案
\begin{enumerate}
    \item [$\bm{a.}$] Every k-length sublist have the running time $\Theta(k^{2})$, in terms of $\Theta$-notation. As a result, $\frac{n}{k}$ sublists have total running time of $\Theta(nk)$.
    \item [$\bm{b.}$]  The recursion tree has $\lg (\frac{n}{k})$ levels, while the cost of each level equals to the total length of the array, which is n. So we can merge the sublists in $\Theta(n\lg(\frac{n}{k}))$ worst-case time.
    \item [$\bm{c.}$]  $k = \Theta(\lg(n))$.
    \item [$\bm{d.}$]  We can choose k as an integer which is the most closet to $\lg(n)$.
\end{enumerate}
\end{solution}
%%%%%%%%%%%%%%%

%%%%%%%%%%%%%%%
\begin{problem}[TC Problem $2-2$]
\end{problem}

\begin{solution}
\begin{enumerate}
    \item [$\bm{a.}$] 
    \item [$\bm{b.}$] 
    \item [$\bm{c.}$] 
    \item [$\bm{d.}$] 
\end{enumerate}
\end{solution}
%%%%%%%%%%%%%%%

%%%%%%%%%%%%%%%
\begin{problem}[TC Problem $2-3$]
\end{problem}

\begin{solution}
\begin{enumerate}
    \item [$\bm{a.}$] $\Theta(n)$.
    \item [$\bm{b.}$] $y = 0$\par
    $\bm{for}$ $i = 0$ $\bm{to}$ $n$ \par
    \hspace{2em} $y = y $+  $a_ix^{i}$\par
    The running time of this algorithm is $\Theta(n^{2})$ ,the efficiency of the algorithm is less than Horner's rule. 
    \item [$\bm{c.}$] $y = x\left(\sum_{k=0}^{n-(i+1)}{a_{k+i+1}x^{k}}\right)+a_i = \left(\sum_{k=1}^{n-i}{a_{k+i}x^{k}}\right)+a_{i} = \sum_{k=0}^{n-i}{a_{k+i}x^{k}}$.
    When it is at termination, which means $i=0$, we have $y=\sum_{k=0}^{n}{a_k}x^{k}$.
    \item [$\bm{d.}$] 
\end{enumerate}
\end{solution}
%%%%%%%%%%%%%%%

%%%%%%%%%%%%%%%
\begin{problem}[TC Problem $2-4$]
\end{problem}

\begin{solution}
\begin{enumerate}
    \item [$\bm{a.}$] $pair(1,5)\quad pair(2,5)\quad pair(3,5)\quad pair(4,5)\quad pair(3,4)$
    \item [$\bm{b.}$]  $<n,n-1,n-2,...,1>$ has the most inversions. It has $\frac{n(n-1)}{2}$ inversions.
    \item [$\bm{c.}$] Linear. Every time we compare the key with the elements on the left of the key, the inversion's number decresed by 1. As a result, the number of inversions correspond to the times of comparision.
    \item [$\bm{d.}$] 
\end{enumerate}
\end{solution}
%%%%%%%%%%%%%%%

%%%%%%%%%%%%%%%
\begin{problem}[TC Problem $3-2$]
\end{problem}

\begin{solution}
  \begin{tabular}[t]{cc|c|c|c|c|c|}
    $\bm{A}$ & $\bm{B}$ & $\bm{O}$ & $\bm{o}$ & $\bm{\Omega}$ & $\bm{\omega}$ & $\bm{\Theta}$\\
    \hline
    $\lg^{k}n$ & $n^{\epsilon}$ &yes &yes &no &no &no\\
    \hline
    $n^{k}$ & $c^{n} $ &yes &yes& no& no&no\\
    \hline
    $\sqrt{n}$ & $n^{sin(n)}$ &no &no &no &no &no\\
    \hline 
    $2^{n} $& $2^{\frac{n}{2}}$ & no &no &yes &yes &no \\
    \hline
    $n^{\lg c}$ & $c^{\lg n}$ &yes &no &yes &no &yes\\
    \hline
    $\lg{(n!)}$ & $\lg(n^{n})$ & yes &no &yes &no &yes\\
    \hline

\end{tabular}
\end{solution}
%%%%%%%%%%%%%%%


%%%%%%%%%%%%%%%
\begin{problem}[TC Problem $3-4$]
\end{problem}

\begin{solution}
\begin{enumerate}
    \item [$\bm{a.}$]False, we can choose  $f(n) = n$, $g(n) = n^{2}$.
    \item [$\bm{b.}$]False, we can choose $f(n) = n$, $g(n) = n^{2}$.
    \item [$\bm{c.}$]If $f(n) = O(g(n))$, there exist a positive constants $c$ and $n_0$ such that $1 \le f(n) \le cg(n)$ for all $n \ge n_0$, thus $\lg(f(n)) \le$ $\lg(cg(n))$= $\lg(c)$ + $\lg(g(n))$=O($\lg(g(n))$).
    \item [$\bm{d.}$]Just as the problem above, $0 \le f(n) \le cg(n)$, thus $2^{f(n)} \le 2^{cg(n)} = 2^{c} \cdot 2^{g(n)} = O(2^{g(n)})$.
    \item [$\bm{e.}$] As we suppose that when n increase to infinity, $f(n)$ must be unbounded, which implies $f(n) \ge 1$, thus $f(n) \le f(n)^{2}$.
    \item [$\bm{f.}$] There exist a positive constants $c$ and $n_0$ such that $0 \le f(n) \le cg(n)$ for all $n \ge n_0$, so we can choose $\frac{1}{c}$ and the $n_0$ such that $\frac{1}{c}f(n) \le g(n)$ for all $n \ge n_0$.
    \item [$\bm{g.}$] False, we can choose $f(n) = 2^{n}$.
    \item[$\bm{h.}$]We let $g(n) = o(f(n))$, thus for any positive constant $c > 0$, there exists a constant $n_0 > 0$ such that $0 \le g(n) < cf(n) $ for all $n \ge n_0$, which implies $f(n) \le f(n) + g(n) \le (1+c)f(n) $ , thus $f(n) + o(f(n)$ = $\Theta(f(n))$.
\end{enumerate} 
\end{solution}
%%%%%%%%%%%%%%%


% 如果需要反馈你可以使用:
%\beginfb 
% \begin{itemize}
%   \item 对课程及教师的建议与意见
%   \item 教材中不理解的内容
%   \item 希望深入了解的内容
%   \item $\cdots$
% \end{itemize}
\end{document}
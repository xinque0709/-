%2-1-correctness.tex

\documentclass[a4paper, justified]{tufte-handout}
\input{hw-preamble} % feel free to modify this file
\title{第1讲: 算法问题与解题的正确性}
\me{林方浩}{231098078}{}{}
\date{截止时间:2024年3月3日} % or like 2019年9月13日
%\newcommand{\lg}{\log_{10}}
\usepackage{bm}
\begin{document}
\maketitle
\noplagiarism % always keep this line

\beginrequired

%%%%%%%%%%%%%%%
\begin{problem}[TC Problem $2-1$]
\end{problem}

\begin{solution}
% 填写你的答案
\begin{enumerate}
    \item [$\bm{a.}$] Every k-length sublist have the running time $\Theta(k^{2})$, in terms of $\Theta$-notation. As a result, $\frac{n}{k}$ sublists have total running time of $\Theta(nk)$.
    \item [$\bm{b.}$]  The recursion tree has $\lg (\frac{n}{k})$ levels, while the cost of each level equals to the total length of the array, which is n. So we can merge the sublists in $\Theta(n\lg(\frac{n}{k}))$ worst-case time.
    \item [$\bm{c.}$]  $k = \Theta(\lg(n))$.
    \item [$\bm{d.}$]  We can choose k as an integer which is the most closet to $\lg(n)$.
\end{enumerate}
\end{solution}
%%%%%%%%%%%%%%%

%%%%%%%%%%%%%%%
\begin{problem}[TC Problem $2-2$]
\end{problem}

\begin{solution}
\begin{enumerate}
    \item [$\bm{a.}$] We alse need to prove that : the set of elements in $A'$ is the same as the set of elements in $A$.
    \item [$\bm{b.}$] The loop invariant is :$A[j]$ is the smallest element in the array$A[j..n]$.\\ $\bm{Initialization:}$ $j=A.length$, then $A[j..n]$ is just $A[n]$, it is obviously right. \\$\bm{Maintenance:}$ If $A[j] < A[j-1]$,then after exchange them, the number indexed by $j-1$ is the smallest element in $A[j-1..n]$ since all elements in $A[j+1..n]$ is greater than $A[j]$. If $A[j] \ge A[j-1]$ , then the conclusion does not change. \\$\bm{Termination:}$ We have $j=i$, according to the maintenance part, $A[i]$ is the smallest element in the array $A[i..n]$.
    \item [$\bm{c.}$] The loop invariant is : The array $A[1..i]$ is sorted.\\ $\bm{Initialization:}$ We have $i=1$, according to the termination in part $\bm{b}$ ,it is correct.\\ $\bm{Maintenance:}$ Since we have already proved that $A[i+1]$ is the smallest element in the array $A[i+1..n]$, and we suppose that the array $A[1..i]$ is sorted, thus $A[1..i+1]$ is sorted.\\ $\bm{Termination:}$ This time, i changes to $A.length$,which implies that the whole array is sorted.
    \item [$\bm{d.}$] The worst-case running time of bubblesort is $\Theta(n^{2})$, which is the same as the insertion sort.
\end{enumerate}
\end{solution}
%%%%%%%%%%%%%%%

%%%%%%%%%%%%%%%
\begin{problem}[TC Problem $2-3$]
\end{problem}

\begin{solution}
\begin{enumerate}
    \item [$\bm{a.}$] $\Theta(n)$.
    \item [$\bm{b.}$] $y = 0$\par
    $\bm{for}$ $i = 0$ $\bm{to}$ $n$ \par
    \hspace{2em} $y = y $+  $a_ix^{i}$\par
    The running time of this algorithm is $\Theta(n^{2})$ ,the efficiency of the algorithm is less than Horner's rule. 
    \item [$\bm{c.}$] $y = x\left(\sum_{k=0}^{n-(i+1)}{a_{k+i+1}x^{k}}\right)+a_i = \left(\sum_{k=1}^{n-i}{a_{k+i}x^{k}}\right)+a_{i} = \sum_{k=0}^{n-i}{a_{k+i}x^{k}}$.
    When it is at termination, which means $i=0$, we have $y=\sum_{k=0}^{n}{a_k}x^{k}$.
    \item [$\bm{d.}$] According to part $\bm{c}$, it is correct when it terminates.
\end{enumerate}
\end{solution}
%%%%%%%%%%%%%%%

%%%%%%%%%%%%%%%
\begin{problem}[TC Problem $2-4$]
\end{problem}

\begin{solution}
\begin{enumerate}
    \item [$\bm{a.}$] $pair(1,5)\quad pair(2,5)\quad pair(3,5)\quad pair(4,5)\quad pair(3,4)$
    \item [$\bm{b.}$]  $<n,n-1,n-2,...,1>$ has the most inversions. It has $\frac{n(n-1)}{2}$ inversions.
    \item [$\bm{c.}$] Linear. Every time we compare the key with the elements on the left of the key, the inversion's number decresed by 1. As a result, the number of inversions correspond to the times of comparision.
    \item [$\bm{d.}$] By using the algorithm similar to the merge sort, we can count the inversion numbers in every sublists, and keep track on the inversions when merging the two sublists, we sum the two parts ,that is the answer.
\end{enumerate}
\end{solution}
%%%%%%%%%%%%%%%

%%%%%%%%%%%%%%%
\begin{problem}[TC Problem $3-2$]
\end{problem}

\begin{solution}
  \begin{tabular}[t]{cc|c|c|c|c|c|}
    $\bm{A}$ & $\bm{B}$ & $\bm{O}$ & $\bm{o}$ & $\bm{\Omega}$ & $\bm{\omega}$ & $\bm{\Theta}$\\
    \hline
    $\lg^{k}n$ & $n^{\epsilon}$ &yes &yes &no &no &no\\
    \hline
    $n^{k}$ & $c^{n} $ &yes &yes& no& no&no\\
    \hline
    $\sqrt{n}$ & $n^{sin(n)}$ &no &no &no &no &no\\
    \hline 
    $2^{n} $& $2^{\frac{n}{2}}$ & no &no &yes &yes &no \\
    \hline
    $n^{\lg c}$ & $c^{\lg n}$ &yes &no &yes &no &yes\\
    \hline
    $\lg{(n!)}$ & $\lg(n^{n})$ & yes &no &yes &no &yes\\
    \hline

\end{tabular}
\end{solution}
%%%%%%%%%%%%%%%


%%%%%%%%%%%%%%%
\begin{problem}[TC Problem $3-4$]
\end{problem}

\begin{solution}
\begin{enumerate}
    \item [$\bm{a.}$]False, we can choose  $f(n) = n$, $g(n) = n^{2}$.
    \item [$\bm{b.}$]False, we can choose $f(n) = n$, $g(n) = n^{2}$.
    \item [$\bm{c.}$]If $f(n) = O(g(n))$, there exist a positive constants $c$ and $n_0$ such that $1 \le f(n) \le cg(n)$ for all $n \ge n_0$, thus $\lg(f(n)) \le$ $\lg(cg(n))$= $\lg(c)$ + $\lg(g(n))$=O($\lg(g(n))$).
    \item [$\bm{d.}$]Just as the problem above, $0 \le f(n) \le cg(n)$, thus $2^{f(n)} \le 2^{cg(n)} = 2^{c} \cdot 2^{g(n)} = O(2^{g(n)})$.
    \item [$\bm{e.}$] As we suppose that when n increase to infinity, $f(n)$ must be unbounded, which implies $f(n) \ge 1$, thus $f(n) \le f(n)^{2}$.
    \item [$\bm{f.}$] There exist a positive constants $c$ and $n_0$ such that $0 \le f(n) \le cg(n)$ for all $n \ge n_0$, so we can choose $\frac{1}{c}$ and the $n_0$ such that $\frac{1}{c}f(n) \le g(n)$ for all $n \ge n_0$.
    \item [$\bm{g.}$] False, we can choose $f(n) = 2^{n}$.
    \item[$\bm{h.}$]We let $g(n) = o(f(n))$, thus for any positive constant $c > 0$, there exists a constant $n_0 > 0$ such that $0 \le g(n) < cf(n) $ for all $n \ge n_0$, which implies $f(n) \le f(n) + g(n) \le (1+c)f(n) $ , thus $f(n) + o(f(n)$ = $\Theta(f(n))$.
\end{enumerate} 
\end{solution}
%%%%%%%%%%%%%%%


% 如果需要反馈你可以使用:
%\beginfb 
% \begin{itemize}
%   \item 对课程及教师的建议与意见
%   \item 教材中不理解的内容
%   \item 希望深入了解的内容
%   \item $\cdots$
% \end{itemize}
\end{document}